Prendendo come riferimento il formato CRS per il calcolo del prodotto tra una matrice sparsa ed un vettore e considerando il tool di sintesi ad alto livello per sistemi digitali, fornito da {Xilinx\textsuperscript{\textregistered} Vivado\textsuperscript{\textregistered}, analizzare le soluzioni proposte nella seguente tabella utilizzando le direttive proprietarie citate e caratterizzando in termini di latenza, dissipazione di potenza e utilizzazione delle risorse.

\begin{table}[H]
	\centering
	\begin{tabular}{|c|c|c|}
		\hline
		\textbf{Solution} & Loop1 & Loop2 \\
		\hline
		1 & - & - \\
		\hline
		2 & - & Pipeline \\
		\hline
		3 & Pipeline & - \\
		\hline
		4 & Unroll=2 & - \\
		\hline
		5 & - & Pipeline, Unroll=2 \\
		\hline
		6 & - & Pipeline, Unroll=2, Cyclic=2 \\
		\hline
		7 & - & Pipeline, Unroll=4 \\
		\hline
		8 & - & Pipeline, Unroll=2, Cyclic=4 \\
		\hline
		9 & - & Pipeline, Unroll=8 \\
		\hline
		10 & - & Pipeline, Unroll=2, Cyclic=8 \\
		\hline
		11 & - & Pipeline, Unroll=2, Block=8 \\
		\hline
	\end{tabular}
	\caption{SMVM Solutions To Be Performed}
	\label{tab:smvm-solutions-to-be-performed}
\end{table}