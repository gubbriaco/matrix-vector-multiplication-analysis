Nell'analisi numerica, una \textbf{matrice sparsa} è una matrice in cui la maggior parte degli elementi presenta un valore nullo. Non esiste una definizione rigorosa della proporzione di elementi a valore nullo affinché una matrice possa essere considerata sparsa. Al contrario, se la maggior parte degli elementi è non nulla, allora la matrice è considerata densa.
\\
Una matrice è tipicamente memorizzata come un array bidimensionale. Ogni voce della matrice rappresenta un elemento $a_{i,j}$ e vi si accede tramite l'indice di riga i e l'indice di colonna j. Per una matrice m × n, la quantità di memoria necessaria per memorizzare la matrice in questo formato è proporzionale a m × n (senza considerare che è necessario memorizzare anche le dimensioni relative alla matrice).
\\
Nel caso di una matrice sparsa, è possibile ridurre notevolmente i requisiti di memoria memorizzando solo le voci non nulle. A seconda del numero e della distribuzione delle voci non nulle, è possibile utilizzare diverse strutture di dati che consentono di ottenere enormi risparmi di memoria rispetto all'approccio di base. Il compromesso è che l'accesso ai singoli elementi diventa più complesso e sono necessarie strutture aggiuntive per poter recuperare la matrice originale senza ambiguità.
\\
I formati possono essere divisi in due gruppi:
\begin{itemize}
	\item Quelli che supportano una modifica efficiente, come DOK (Dictionary of Keys), LIL (List of Lists) o COO (Coordinate List), utilizzati solitamente per la costruzione della matrice.
	\item Quelli che supportano l'accesso e le operazioni matriciali efficienti, come CRS (Compressed Row Storage) o CCS (Compressed Column Storage).
\end{itemize}
