Qui, di seguito, vengono riportate le architetture relative alla sesta solution. In particolare, come già precedentemente citato, tale solution prevede l'utilizzo della direttiva di partitioning. Nello specifico, verranno analizzate le seguenti implementazioni relative al loop2:
\begin{itemize}
	\item Pipeline, Unroll=2, Cyclic=2 (columnIndex, values, x)
	\item Pipeline, Unroll=2, Cyclic=2 (columnIndex)
	\item Pipeline, Unroll=2, Cyclic=2 (values)
	\item Pipeline, Unroll=2, Cyclic=2 (x)
\end{itemize}

In particolare, è possibile evidenziare nel dettaglio le differenti soluzioni hardware nei seguenti allegati.
\lstinputlisting[language=C++]{solutions/s6/s6all3.cpp}
\lstinputlisting[language=C++]{solutions/s6/s6columnIndex.cpp}
\lstinputlisting[language=C++]{solutions/s6/s6values.cpp}
\lstinputlisting[language=C++]{solutions/s6/s6x.cpp}

Effettuando la sintesi è possibile evidenziare il seguente report:\\

\begin{table}[H]
	\centering
	\begin{tabular}{|c|c|c|c|c|}
		\hline
		\textbf{Solution} & \textbf{Clock} & \textbf{Target} & \textbf{Estimated} & \textbf{Uncertainty} \\
		\hline
		columnIndex, values, x & ap\_clk & 10.00 & 8.510 & 1.25 \\
		\hline
		columnIndex & ap\_clk & 10.00 & 8.510 & 1.25 \\
		\hline
		values & ap\_clk & 10.00 & 8.510 & 1.25 \\
		\hline
		x & ap\_clk & 10.00 & 8.510 & 1.25 \\
		\hline
	\end{tabular}
	\caption{HLS Solution 6 Timing Summary (ns)}
	\label{tab:hls-solution-6-timing-summary}
\end{table}

\begin{table}[H]
	\centering
	\begin{tabular}{|c|c|c|c|c|}
		\hline
		\multicolumn{1}{|c|}{\textbf{Solution}} & \multicolumn{2}{|c|}{\textbf{Latency}} & \multicolumn{2}{|c|}{\textbf{Interval}} \\
		& min & max & min & max \\
		\hline
		columnIndex, values, x & 33 & 41 & 33 & 41 \\
		\hline
		columnIndex & 56 & 56 & 56 & 56 \\
		\hline
		values & 61 & 66 & 61 & 66 \\
		\hline
		x & 61 & 66 & 61 & 66 \\
		\hline
	\end{tabular}
	\caption{HLS Solution 6 Latency Summary (clock cycles)}
	\label{tab:hls-solution-6-latency-summary}
\end{table}

\begin{table}[H]
	\centering
	\begin{tabular}{|c|c|c|c|c|c|c|c|c|c|}
		\hline
		\multicolumn{1}{|c|}{\textbf{Solution}} & \multicolumn{1}{|c|}{Loop Name} & \multicolumn{2}{|c|}{\textbf{Latency}} & \multicolumn{1}{c|}{\textbf{Iteration Latency}} & \multicolumn{2}{c|}{\textbf{Initiation Interval}} & \multicolumn{1}{c|}{\textbf{Trip}}  \\
		&  & min & max & & achieved & target & \textbf{Count} \\
		\hline
		columnIndex, values, x & - loop1 & 32 & 40 & 8$\sim$10 & - & - & 4 \\
		& + loop2 & 4 & 6 & 5 & 1 & 1 & 0$\sim$2 \\
		\hline
		columnIndex & - loop1 & 10 & 10 & 2 & - & - & 5 \\
		& + loop2 & 44 & 44 & 4 & - & - & 11 \\
		\hline
		values & - loop1 & 10 & 10 & 2 & - & - & 5 \\
		& + loop2 & 44 & 44 & 4 & - & - & 11 \\
		\hline
		x & - loop1 & 10 & 10 & 2 & - & - & 5 \\
		& + loop2 & 44 & 44 & 4 & - & - & 11 \\
		\hline
	\end{tabular}
	\caption{HLS Solution 6 Latency Loops Summary }
	\label{tab:hls-solution-6-loop-summary}
\end{table}

\begin{table}[H]
	\centering
	\begin{tabular}{|c|c|c|c|c|}
		\hline
		\textbf{Solution} & \textbf{BRAM\_18K} & \textbf{DSP48E} & \textbf{FF} & \textbf{LUT} \\
		\hline
		columnIndex, values, x & 0 & 6 & 535 & 623 \\
		\hline
		columnIndex & 2 & 2 & 145 & 236 \\
		\hline
		values & 2 & 2 & 145 & 236 \\
		\hline
		x & 2 & 2 & 145 & 236 \\
		\hline
	\end{tabular}
	\caption{HLS Solution 6 Utilization Estimates [\#]}
	\label{tab:hls-solution-6-utilization-report}
\end{table}

\begin{table}[H]
	\centering
	\begin{tabular}{|c|c|c|c|c|c|c|c|c|}
		\hline
		\multicolumn{1}{|c|}{\textbf{Solution}} & \multicolumn{1}{|c|}{RTL} & \multicolumn{1}{|c|}{Status} & \multicolumn{3}{c|}{\textbf{Latency}} & \multicolumn{3}{c|}{\textbf{Interval}} \\
		& &  & min & avg & max & min & avg & max \\
		\hline
		columnIndex, values, x & VHDL & Pass & 37 & 37 & 37 & NA & NA & NA \\
		\hline
		columnIndex & VHDL & Pass & 56 & 56 & 57 & NA & NA & NA \\
		\hline
		values & VHDL & Pass & 56 & 56 & 57 & NA & NA & NA \\
		\hline
		x & VHDL & Pass & 56 & 56 & 57 & NA & NA & NA \\
		\hline
	\end{tabular}
	\caption{HLS Solution 6 C/RTL Cosimulation Report }
	\label{tab:hls-solution-6-cosimulation-report}
\end{table}

\begin{table}[H]
	\centering
	\begin{tabular}{|c|c|c|c|c|c|c|c|c|}
		\hline
		\textbf{Solution} & \textbf{SLICE} & \textbf{LUT} & \textbf{FF} & \textbf{DSP} & \textbf{BRAM} & \textbf{CP} & \textbf{CP} & \textbf{CP} \\
		& & & & & & \textbf{required} & \textbf{achieved} & \textbf{achieved}\\
		& & & & & & & \textbf{post-} & \textbf{post-}\\
		& & & & & & & \textbf{synthesis} & \textbf{implementation}\\
		\hline
		columnIndex, values, x  & 113 & 316 & 198 & 6 & 0 & 10 & 7.927 & 7.799 \\
		\hline
		columnIndex  & 31 & 97 & 72 & 2 & 2 & 10 & 5.745 & 5.692 \\
		\hline
		values  & 31 & 97 & 72 & 2 & 2 & 10 & 5.745 & 5.692 \\
		\hline
		x  & 31 & 97 & 72 & 2 & 2 & 10 & 5.745 & 5.692 \\
		\hline
	\end{tabular}
	\caption{HLS Loop Unrolling Factor=2 Solution Export RTL Report}
	\label{tab:hls-solution-6-export-rtl-report}
\end{table}