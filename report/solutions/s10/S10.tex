Nella soluzione hardware in questione verrà utilizzata la direttiva di pipeline, di unrolling di fattore pari a 2 e la direttiva di partizionamento di tipologia cyclic. Nello specifico, verranno analizzate le seguenti implementazioni relative al loop2:
\begin{itemize}
	\item Pipeline, Unroll=2, Cyclic=8 (columnIndex, values, x)
	\item Pipeline, Unroll=2, Cyclic=8 (columnIndex)
	\item Pipeline, Unroll=2, Cyclic=8 (values)
	\item Pipeline, Unroll=2, Cyclic=8 (x)
\end{itemize}

In particolare, è possibile evidenziare nel dettaglio le differenti soluzioni hardware nei seguenti allegati.
\lstinputlisting[language=C++]{solutions/s10/s10all3.cpp}
\lstinputlisting[language=C++]{solutions/s10/s10columnIndex.cpp}
\lstinputlisting[language=C++]{solutions/s10/s10values.cpp}
\lstinputlisting[language=C++]{solutions/s10/s10x.cpp}

Effettuando la sintesi si ottiene il seguente log.
\\
\textcolor{red}{ERROR: [XFORM 203-103] Cannot partition array 'x' (smvmProject/smvm.cpp:11): incorrect partition factor 8.}
\\
In particolare, la console sta segnalando che effettivamente non riesce a partizionare l'array x dal momento che la dimensione dell'array risulta essere non compatibile con il fattore di partitioning dichiarato. Infatti, la dimensione di x inizialmente dichiarata in definitions.h è pari a 4 mentre il fattore di partizionamento che si sta utilizzando è pari a 8. A questo proposito si potrebbe modificare il valore di dimensionamento relativo a x all'interno dell'header. In particolare è necessario modificare sia il valore di size sia il valore di rows poiché l'applicazione che si sta implementando richiede che la matrice sia quadrata. Inoltre, bisogna anche modificare i parametri dichiarati all'interno della direttiva trip count dal momento che il numero di righe e, quindi, di iterazioni risulta essere differente.
\lstinputlisting[language=C]{solutions/s10/headermodified.h}
\lstinputlisting[language=C++]{solutions/s10/s10all3modified.cpp}
\lstinputlisting[language=C++]{solutions/s10/s10columnIndexmodified.cpp}
\lstinputlisting[language=C++]{solutions/s10/s10valuesmodified.cpp}
\lstinputlisting[language=C++]{solutions/s10/s10xmodified.cpp}

Pertanto, effettuando la sintesi si ottiene il seguente report.

\begin{table}[H]
	\centering
	\begin{tabular}{|c|c|c|c|c|}
		\hline
		\textbf{Solution} & \textbf{Clock} & \textbf{Target} & \textbf{Estimated} & \textbf{Uncertainty} \\
		\hline
		columnIndex, values, x & ap\_clk & 10.00 & 8.510 & 1.25 \\
		\hline
		columnIndex & ap\_clk & 10.00 & 8.510 & 1.25 \\
		\hline
		values & ap\_clk & 10.00 & 8.510 & 1.25 \\
		\hline
		x & ap\_clk & 10.00 & 8.510 & 1.25 \\
		\hline
	\end{tabular}
	\caption{HLS Solution 10 Timing Summary (ns)}
	\label{tab:hls-solution-10-timing-summary}
\end{table}

\begin{table}[H]
	\centering
	\begin{tabular}{|c|c|c|c|c|}
		\hline
		\multicolumn{1}{|c|}{\textbf{Solution}} & \multicolumn{2}{|c|}{\textbf{Latency}} & \multicolumn{2}{|c|}{\textbf{Interval}} \\
		& min & max & min & max \\
		\hline
		columnIndex, values, x & 57 & 89 & 57 & 89 \\
		\hline
		columnIndex & 65 & 97 & 65 & 97 \\
		\hline
		values & 65 & 97 & 65 & 97 \\
		\hline
		x & 57 & 89 & 57 & 89 \\
		\hline
	\end{tabular}
	\caption{HLS Solution 10 Latency Summary (clock cycles)}
	\label{tab:hls-solution-10-latency-summary}
\end{table}

\begin{table}[H]
	\centering
	\begin{tabular}{|c|c|c|c|c|c|c|c|c|c|}
		\hline
		\multicolumn{1}{|c|}{\textbf{Solution}} & \multicolumn{1}{|c|}{Loop Name} & \multicolumn{2}{|c|}{\textbf{Latency}} & \multicolumn{1}{c|}{\textbf{Iteration Latency}} & \multicolumn{2}{c|}{\textbf{Initiation Interval}} & \multicolumn{1}{c|}{\textbf{Trip}}  \\
		&  & min & max & & achieved & target & \textbf{Count} \\
		\hline
		columnIndex, values, x & - loop1 & 56 & 88 & 7$\sim$11 & - & - & 8 \\
		& + loop2 & 3 & 7 & 4 & 1 & 1 & 0$\sim$4 \\
		\hline
		columnIndex & - loop1 & 64 & 96 & 8$\sim$12 & - & - & 8 \\
		& + loop2 & 4 & 8 & 5 & 1 & 1 & 0$\sim$4 \\
		\hline
		values & - loop1 & 64 & 96 & 8$\sim$12 & - & - & 8 \\
		& + loop2 & 4 & 8 & 5 & 1 & 1 & 0$\sim$4 \\
		\hline
		x & - loop1 & 56 & 88 & 7$\sim$11 & - & - & 8 \\
		& + loop2 & 3 & 7 & 4 & 1 & 1 & 0$\sim$4 \\
		\hline
	\end{tabular}
	\caption{HLS Solution 10 Latency Loops Summary }
	\label{tab:hls-solution-10-loop-summary}
\end{table}

\begin{table}[H]
	\centering
	\begin{tabular}{|c|c|c|c|c|}
		\hline
		\textbf{Solution} & \textbf{BRAM\_18K} & \textbf{DSP48E} & \textbf{FF} & \textbf{LUT} \\
		\hline
		columnIndex, values, x & 0 & 6 & 762 & 900 \\
		\hline
		columnIndex & 0 & 6 & 539 & 752 \\
		\hline
		values & 0 & 6 & 640 & 773 \\
		\hline
		x & 0 & 6 & 727 & 492 \\
		\hline
	\end{tabular}
	\caption{HLS Solution 10 Utilization Estimates [\#]}
	\label{tab:hls-solution-10-utilization-report}
\end{table}

In particolare, dall'interfaccia Analysis sotto allegata (corrispondente al partitioning dell'array x), si può notare come l'unrolling di fattore 2 sia stato applicato correttamente dal momento che vengono effettuate 2 operazioni di moltiplicazione e 2 operazioni di somma in un'unica iterazione del loop2. Nello specifico, si può evidenziare come i prodotti richiedano l'utilizzo di 3 DSP ognuno tale da giustificare l'utilizzazione di tali risorse pari a 6 come riportato nel report di sintesi. Inoltre, si può notare come il partitioning di fattore pari a 8 abbia effettuato il partizionamento dell'array x in 8 sub-array. In particolare, ogni sub-array prevede un'utilizzazione di bit pari a 32 e una corrispondente utilizzazione dei FF pari a 32 ognuno. Questo aspetto è di fondamentale importanza poiché dal momento che l'array x presenta dimensione pari a 8 e dal momento che il fattore di partizionamento è pari a 8, questo vuol dire che il tool ha effettuato un partitioning di tipo cyclic corrispondente ad un partitioning di tipo complete, cioè il risultato sono dei sub-array di dimensione pari a 1 (poiché la dimensione di x coincide con il fattore di partizionamento). Tanto è vero che effettuando l'array partitioning di tipologia complete sul solo array x, si avrebbe la stessa utilizzazione di risorse e, soprattutto, l'interfaccia Analysis e Resource Profile corrisponderebbero a quelle sotto allegate. Infatti, bisogna ricordare che l'array partitioning di tipologia complete comporta che l'array venga suddiviso in singoli registri mentre quello di tipologia cyclic comporta che vengano creati blocchi ciclici di uguali dimensioni interlacciando gli elementi dell'array iniziale. Pertanto, ciò che cambia tra le due tipologie è che il secondo vada ad assegnare in questi sub-array in maniera ciclica gli elementi dell'array iniziale in base al fattore scelto. Però, nel caso in cui tale fattore corrisponde alla dimensione dell'array iniziale, nel momento in cui il tool assegna l'ultimo elemento ad un sub-array (dopodiché dovrebbe assegnare il prossimo elemento se presente nell'array, ma non in questo caso, al primo sub-array e così via) terminando la procedura di partitioning e lasciando in ogni sub-array un solo elemento. Questo sostanzialmente corrisponderebbe ad avere 8 registri a 32 bit ognuno come succederebbe nella tipologia complete. Invece, per quanto riguarda gli array columnIndex e values, dal momento che presentano dimensione differente rispetto al fattore di partizionamento, questo si traduce in differenti scheduling e performance.

\begin{figure}[H]
	\centering
	\includegraphics[width=1\textwidth]{solutions/s10/s10analysis.png}
	\caption{HLS Solution 10 Analysis}
\end{figure}

\begin{table}[H]
	\centering
	\begin{tabular}{|c|c|c|c|c|c|c|c|c|}
		\hline
		\multicolumn{1}{|c|}{\textbf{Solution}} & \multicolumn{1}{|c|}{RTL} & \multicolumn{1}{|c|}{Status} & \multicolumn{3}{c|}{\textbf{Latency}} & \multicolumn{3}{c|}{\textbf{Interval}} \\
		& &  & min & avg & max & min & avg & max \\
		\hline
		columnIndex, values, x & VHDL & Pass & 61 & 61 & 61 & NA & NA & NA \\
		\hline
		columnIndex & VHDL & Pass & 69 & 69 & 69 & NA & NA & NA \\
		\hline
		values & VHDL & Pass & 69 & 69 & 69 & NA & NA & NA \\
		\hline
		x & VHDL & Pass & 61 & 61 & 61 & NA & NA & NA \\
		\hline
	\end{tabular}
	\caption{HLS Solution 10 C/RTL Cosimulation Report }
	\label{tab:hls-solution-10-cosimulation-report}
\end{table}

Si può notare come l'utilizzazione delle risorse sia aumentata rispetto alle soluzioni 6 (loop2 Pipeline, Unroll=2, Cyclic=2) e 8 (loop2 Pipeline, Unroll=2, Cyclic=4). In particolare, considerando le soluzioni hardware corrispondenti al partizionamento di tutti e tre gli array, si evidenzia rispettivamente un aumento di circa il $66\%$ in corrispondenza delle slice rispetto alla soluzione 6 e 8, un aumento di circa il $77\%$ e di circa il $63\%$ in corrispondenza delle LUT, un aumento di circa il $127\%$ e di circa il $43\%$. 
\\
Per quanto riguarda le soluzioni hardware corrispondenti al partizionamento di columnIndex, rispetto alla soluzione 6 e 8, si registra rispettivamente un aumento di un'unità e una diminuzione di 2 slice, un aumento di nove unità e di 7 unità delle LUT, un aumento di 6 unità e di 4 unità dei FF. 
\\
Per quanto riguarda le solution corrispondenti al partitioning di values, rispetto alla soluzione 6 e 8, si registra rispettivamente un aumento di circa il $18\%$ e di circa il $27\%$ delle slice, un aumento di 15 unità e di circa il $25\%$ delle LUT, un aumento di 25 unità e di 3 unità dei FF. 
\\
Per quanto riguarda le soluzioni hardware corrispondenti al partizionamento di x, rispetto alla soluzione 6 e 8, si registra rispettivamente un aumento di circa il $44\%$ e di circa il $38\%$ delle slice, un aumento di circa il $24\%$ e di circa il $25\%$ delle LUT, un aumento di circa il $124\%$ e di circa il $42\%$ dei FF.

\begin{table}[H]
	\centering
	\begin{tabular}{|c|c|c|c|c|c|c|c|c|}
		\hline
		\textbf{Solution} & \textbf{SLICE} & \textbf{LUT} & \textbf{FF} & \textbf{DSP} & \textbf{BRAM} & \textbf{CP} & \textbf{CP} & \textbf{CP} \\
		& & & & & & \textbf{required} & \textbf{achieved} & \textbf{achieved}\\
		& & & & & & & \textbf{post-} & \textbf{post-}\\
		& & & & & & & \textbf{synthesis} & \textbf{implementation}\\
		\hline
		columnIndex, values, x  & 204 & 558 & 449 & 6 & 0 & 10 & 6.540 & 6.840 \\
		\hline
		columnIndex  & 83 & 268 & 230 & 6 & 0 & 10 & 7.496 & 8.115 \\
		\hline
		values  & 117 & 342 & 199 & 6 & 0 & 10 & 7.927 & 7.603 \\
		\hline
		x  & 143 & 311 & 444 & 6 & 0 & 10 & 6.540 & 6.790 \\
		\hline
	\end{tabular}
	\caption{HLS Solution 10 Export RTL Report}
	\label{tab:hls-solution-10-export-rtl-report}
\end{table}

Ovviamente questo aumento di utilizzazione delle risorse è dovuto all'incremento del fattore di partizionamento adottato nelle soluzioni hardware. Infatti, si evidenzia che il maggiore incremento delle risorse si ha rispetto alla soluzione 6 dove era previsto un fattore pari a 2 rispetto alla soluzione in questione dove il fattore di partitioning previsto è stato di 8.