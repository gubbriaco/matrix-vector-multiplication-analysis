Qui, di seguito, viene riportata l'architettura relativa alla terza solution.

\lstinputlisting[language=C++]{solutions/s3/s3.cpp}

In particolare, nella soluzione hardware in questione, rispetto alla solution 1, è stato aggiunta la direttiva di pipeline all'interno del loop1.

Effettuando la sintesi è possibile evidenziare il seguente report:\\

\begin{table}[H]
	\centering
	\begin{minipage}[t]{0.45\linewidth}
		\centering
		\begin{tabular}{|c|c|c|c|}
			\hline
			\textbf{Clock} & \textbf{Target} & \textbf{Estimated} & \textbf{Uncertainty} \\
			\hline
			ap\_clk & 10.00 & 8.510 & 1.25 \\
			\hline
		\end{tabular}
		\caption{HLS Solution 3 Timing Summary (ns)}
		\label{tab:hls-solution-3-timing-summary}
	\end{minipage}
	\hfill
	\begin{minipage}[t]{0.45\linewidth}
		\centering
		\begin{tabular}{|c|c|c|c|}
			\hline
			\multicolumn{2}{|c|}{\textbf{Latency}} & \multicolumn{2}{|c|}{\textbf{Interval}} \\
			min & max & min & max \\
			\hline
			13 & 93 & 13 & 93 \\
			\hline
		\end{tabular}
		\caption{HLS Solution 3 Latency Summary (clock cycles)}
		\label{tab:hls-solution-3-latency-summary}
	\end{minipage}
\end{table}

\begin{table}[H]
	\centering
	\begin{tabular}{|c|c|c|c|c|c|c|c|c|}
		\hline
		\multicolumn{1}{|c|}{Loop} & \multicolumn{2}{|c|}{\textbf{Latency}} & \multicolumn{1}{c|}{\textbf{Iteration Latency}} & \multicolumn{2}{c|}{\textbf{Initiation Interval}} & \multicolumn{1}{c|}{\textbf{Trip Count}}  \\
		Name & min & max &  & achieved & target &  \\
		\hline
		- loop1 & 12 & 92 & 3$\sim$23 & - & - & 4 \\
		+ loop2 & 0 & 20 & 5 & - & - & 0$\sim$4 \\
		\hline
	\end{tabular}
	\caption{HLS Solution 3 Latency Loops Summary}
	\label{tab:hls-solution-3-loop-summary}
\end{table}

Si può notare come, in questo caso l'Initiation Interval sia non specificato nel loop2 dal momento che la direttiva introdotta nella solution 2 è stata eliminata per la soluzione hardware in questione. Molto più importante è che, considerando la direttiva di pipeline definita all'interno del loop1, in corrispondenza dell'Initiation Interval di tale ciclo non è definito alcun valore numerico. Tanto è vero che, analizzando i log della sintesi presenti nella console è possibile identificare il seguente warning.
\\
\textcolor{blue}{WARNING: [XFORM 203-503] Cannot unroll loop 'loop2' (smvmProject/smvm.cpp:21) in function 'smvm' completely: variable loop bound.}
\\
\textcolor{blue}{WARNING: [SCHED 204-65] Unable to satisfy pipeline directive: Loop contains subloop(s) not being unrolled or flattened.}
\\
In particolare, il tool non è riuscito a soddisfare la richiesta di pipeline a causa dei bound non noti e, pertanto, non riuscendo ad effettuare l'automatic unrolling del loop2. Infatti, si può notare come i valori di latenza siano i medesimi di quelli della solution 1. 
\\
Qui di seguito, viene allegato l'utilizzazione delle risorse stimata dal processo di sintesi. Anche in questo caso il numero di risorse è il medesimo di quello ottenuto in corrispondenza della solution 1.
\begin{table}[h]
	\centering
	\begin{tabular}{|l|c|c|c|c|}
		\hline
		\textbf{Name}    & \textbf{BRAM\_18K} & \textbf{DSP48E} & \textbf{FF} & \textbf{LUT} \\ \hline
		DSP              & -                   & -               & -           & -            \\ 
		Expression       & -                   & 3               & 0           & 137          \\ 
		FIFO             & -                   & -               & -           & -            \\ 
		Instance         & -                   & -               & -           & -            \\ 
		Memory           & 0                   & -               & -          & -            \\ 
		Multiplexer      & -                   & -               & -           & 71          \\ 
		Register         & -                   & -               & 241         & -            \\ \hline
		\textbf{Total}   & 0                   & 3               & 241         & 208          \\ \hline
		\textbf{Available} & 280               & 220             & 106400      & 53200        \\ \hline
		\textbf{Utilization (\%)} & 0            & 1               & $\sim$0     & $\sim$0      \\ \hline
	\end{tabular}
	\caption{HLS Solution 3 Utilization Estimates Summary}
	\label{tab:hls-solution-3-utilization-estimates-summary}
\end{table}

Successivamente effettuando la C/RTL Cosimulation e l'Export RTL è possibile evidenziare i seguenti report. Anche in questo caso, sia il report del C/RTL Cosimulation sia quello dell'Export RTL risultano essere i medesimi di quelli della solution 1.
\begin{table}[H]
	\centering
	\begin{tabular}{|c|c|c|c|c|c|c|c|}
		\hline
		\multicolumn{1}{|c|}{RTL} & \multicolumn{1}{|c|}{Status} & \multicolumn{3}{c|}{\textbf{Latency}} & \multicolumn{3}{c|}{\textbf{Interval}} \\
		&  & min & avg & max & min & avg & max \\
		\hline
		VHDL & Pass & 58 & 58 & 58 & NA & NA & NA \\
		\hline
	\end{tabular}
	\caption{HLS Solution 3 C/RTL Cosimulation Summary }
	\label{tab:hls-solution-3-cosimulation-summary}
\end{table}

\begin{table}[H]
	\centering
	\begin{minipage}[t]{0.45\linewidth}
		\centering
		\begin{tabular}{|l|r|}
			\hline
			\textbf{Resource} & \textbf{VHDL} \\
			\hline
			SLICE & 48 \\
			\hline
			LUT & 94 \\
			\hline
			FF & 161 \\
			\hline
			DSP & 3 \\
			\hline
			BRAM & 0 \\
			\hline
			SRL & 0 \\
			\hline
		\end{tabular}
		\caption{HLS Solution 3 Export RTL Resource Usage}
		\label{tab:hls-solution-3-export-rtl-resoruce-usage}
	\end{minipage}
	\hfill
	\begin{minipage}[t]{0.45\linewidth}
		\centering
		\begin{tabular}{|l|r|}
			\hline
			\textbf{Timing} & \textbf{VHDL} \\
			\hline
			CP required & 10.000 \\
			\hline
			CP achieved post-synthesis & 5.745 \\
			\hline
			CP achieved post-implementation & 5.692 \\
			\hline
		\end{tabular}
		\caption{HLS Solution 3 Export RTL Final Timing}
		\label{tab:hls-solution-3-export-rtl-final-timing}
	\end{minipage}
\end{table}
