Qui, di seguito, viene riportata l'architettura relativa alla prima solution. Bisogna precisare che, solo per questa soluzione hardware, per semplicità è stata considerata la prima configurazione, cioè size=4, noZeroEl=9, rows=4.

\lstinputlisting[language=C++]{solutions/s1.cpp}

Effettuando la sintesi è possibile evidenziare il seguente report:\\
\begin{table}[H]
	\centering
	\begin{minipage}[t]{0.45\linewidth}
		\centering
		\begin{tabular}{|c|c|c|c|}
			\hline
			\textbf{Clock} & \textbf{Target} & \textbf{Estimated} & \textbf{Uncertainty} \\
			\hline
			ap\_clk & 10.00 & 8.510 & 1.25 \\
			\hline
		\end{tabular}
		\caption{HLS Solution 1 Timing Summary (ns)}
		\label{tab:hls-solution-1-timing-summary}
	\end{minipage}
	\hfill
	\begin{minipage}[t]{0.45\linewidth}
		\centering
		\begin{tabular}{|c|c|c|c|}
			\hline
			\multicolumn{2}{|c|}{\textbf{Latency}} & \multicolumn{2}{|c|}{\textbf{Interval}} \\
			min & max & min & max \\
			\hline
			13 & 93 & 13 & 93 \\
			\hline
		\end{tabular}
		\caption{HLS Unoptimized Solution Latency Summary (clock cycles)}
		\label{tab:hls-unoptimized-solution-latency-summary}
	\end{minipage}
\end{table}

\begin{table}[H]
	\centering
	\begin{tabular}{|c|c|c|c|c|c|c|c|c|}
		\hline
		\multicolumn{1}{|c|}{Loop} & \multicolumn{2}{|c|}{\textbf{Latency}} & \multicolumn{2}{c|}{\textbf{Iteration Latency}} & \multicolumn{2}{c|}{\textbf{Initiation Interval}} & \multicolumn{1}{c|}{\textbf{Trip Count}}  \\
		Name & min & max & min & max & achieved & target &  \\
		\hline
		- loop1 & 12 & 92 & 3 & 23 & - & - & 4 \\
		+ loop2 & 0 & 20 & 5 & 5 & - & - & 0$\sim$4 \\
		\hline
	\end{tabular}
	\caption{HLS Solution 1 Latency Loops Summary }
	\label{tab:hls-solution-1-loop-summary}
\end{table}

Qui di seguito, viene allegato l'utilizzazione delle risorse stimata dal processo di sintesi.
\begin{table}[h]
	\centering
	\begin{tabular}{|l|c|c|c|c|}
		\hline
		\textbf{Name}    & \textbf{BRAM\_18K} & \textbf{DSP48E} & \textbf{FF} & \textbf{LUT} \\ \hline
		DSP              & -                   & -               & -           & -            \\ 
		Expression       & -                   & 3               & 0           & 137          \\ 
		FIFO             & -                   & -               & -           & -            \\ 
		Instance         & -                   & -               & -           & -            \\ 
		Memory           & 0                   & -               & -          & -            \\ 
		Multiplexer      & -                   & -               & -           & 71          \\ 
		Register         & -                   & -               & 241         & -            \\ \hline
		\textbf{Total}   & 0                   & 3               & 241         & 208          \\ \hline
		\textbf{Available} & 280               & 220             & 106400      & 53200        \\ \hline
		\textbf{Utilization (\%)} & 0            & 1               & $\sim$0     & $\sim$0      \\ \hline
	\end{tabular}
	\caption{HLS Solution 1 Utilization Estimates Summary}
	\label{tab:hls-solution-1-utilization-estimates-summary}
\end{table}

Successivamente effettuando la C/RTL Cosimulation e l'Export RTL è possibile evidenziare i seguenti report.
\begin{table}[H]
	\centering
	\begin{tabular}{|c|c|c|c|c|c|c|c|}
		\hline
		\multicolumn{1}{|c|}{RTL} & \multicolumn{1}{|c|}{Status} & \multicolumn{3}{c|}{\textbf{Latency}} & \multicolumn{3}{c|}{\textbf{Interval}} \\
		&  & min & avg & max & min & avg & max \\
		\hline
		VHDL & Pass & 58 & 58 & 58 & 58 & 58 & 58 \\
		\hline
	\end{tabular}
	\caption{HLS Solution 1 C/RTL Cosimulation Summary }
	\label{tab:hls-solution-1-cosimulation-summary}
\end{table}

\begin{table}[H]
	\centering
	\begin{minipage}[t]{0.45\linewidth}
		\centering
		\begin{tabular}{|l|r|}
			\hline
			\textbf{Resource} & \textbf{VHDL} \\
			\hline
			SLICE & 45 \\
			\hline
			LUT & 93 \\
			\hline
			FF & 161 \\
			\hline
			DSP & 3 \\
			\hline
			BRAM & 0 \\
			\hline
			SRL & 0 \\
			\hline
		\end{tabular}
		\caption{HLS Solution 1 Export RTL Resource Usage}
		\label{tab:hls-solution-1-export-rtl-resoruce-usage}
	\end{minipage}
	\hfill
	\begin{minipage}[t]{0.45\linewidth}
		\centering
		\begin{tabular}{|l|r|}
			\hline
			\textbf{Timing} & \textbf{VHDL} \\
			\hline
			CP required & 10.000 \\
			\hline
			CP achieved post-synthesis & 5.745 \\
			\hline
			CP achieved post-implementation & 5.692 \\
			\hline
		\end{tabular}
		\caption{HLS Solution Export RTL Final Timing}
		\label{tab:hls-solution-1-export-rtl-final-timing}
	\end{minipage}
\end{table}