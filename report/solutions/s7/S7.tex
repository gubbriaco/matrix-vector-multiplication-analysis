Qui, di seguito, viene riportata l'architettura relativa alla settima solution.

\lstinputlisting[language=C++]{solutions/s7/s7.cpp}

In particolare, rispetto alla soluzione hardware 5 dove era stato considerato un parallelismo di fattore pari a 2, in questa solution è stato considerato un unrolling di fattore pari a 4. In particolare, ciò che ci si aspetta è un aumento delle risorse ed eventuali problematiche relative al timing dal momento che il tool deve gestire all'interno del loop2 più accessi in memoria paralleli.
\\
Effettuando la sintesi è possibile evidenziare il seguente log nella console:
\\
\textcolor{blue}{WARNING: [SCHED 204-69] Unable to schedule 'load' operation ('columnIndex\_load\_2', smvmProject/smvm.cpp:34) on array 'columnIndex' due to limited memory ports. Please consider using a memory core with more ports or partitioning the array 'columnIndex'.}
\\
Tale log sta a significare che non riesce a schedulare correttamente, dal punto di vista degli accessi in memoria, la load operation relativa all'array \textit{columnIndex} dato dal numero limitato di porte relative alla memoria.
\\
In particolare, analizzando il report relativo alla sintesi, si può notare come l'Initiation Interval, associato al loop2, raggiunto risulta essere maggiore di quello target.
\\

\begin{table}[H]
	\centering
	\begin{minipage}[t]{0.45\linewidth}
		\centering
		\begin{tabular}{|c|c|c|c|}
			\hline
			\textbf{Clock} & \textbf{Target} & \textbf{Estimated} & \textbf{Uncertainty} \\
			\hline
			ap\_clk & 10.00 & 8.510 & 1.25 \\
			\hline
		\end{tabular}
		\caption{HLS Solution 7 Timing Summary (ns)}
		\label{tab:hls-solution-7-timing-summary}
	\end{minipage}
	\hfill
	\begin{minipage}[t]{0.45\linewidth}
		\centering
		\begin{tabular}{|c|c|c|c|}
			\hline
			\multicolumn{2}{|c|}{\textbf{Latency}} & \multicolumn{2}{|c|}{\textbf{Interval}} \\
			min & max & min & max \\
			\hline
			37 & 45 & 37 & 45 \\
			\hline
		\end{tabular}
		\caption{HLS Solution 7 Latency Summary (clock cycles)}
		\label{tab:hls-solution-7-latency-summary}
	\end{minipage}
\end{table}

\begin{table}[H]
	\centering
	\begin{tabular}{|c|c|c|c|c|c|c|c|c|}
		\hline
		\multicolumn{1}{|c|}{Loop} & \multicolumn{2}{|c|}{\textbf{Latency}} & \multicolumn{1}{c|}{\textbf{Iteration Latency}} & \multicolumn{2}{c|}{\textbf{Initiation Interval}} & \multicolumn{1}{c|}{\textbf{Trip Count}}  \\
		Name & min & max &  & achieved & target &  \\
		\hline
		- loop1 & 36 & 44 & 9$\sim$11 & - & - & 4 \\
		+ loop2 & 5 & 7 & 6 & \textcolor{red}{2} & 1 & 0$\sim$1 \\
		\hline
	\end{tabular}
	\caption{HLS Solution 7 Latency Loops Summary}
	\label{tab:hls-solution-7-loop-summary}
\end{table}

Pertanto, si potrebbe aggiungere una direttiva di partizionamento relativa all'array menzionato all'interno del log, cioè \textit{columnIndex}.

\lstinputlisting[language=C++]{solutions/s7/s7columnIndex.cpp}

Effettuando nuovamente la sintesi, si ottiene il seguente log nella console e i seguenti valori di latenza.
\\
\textcolor{blue}{WARNING: [SCHED 204-69] Unable to schedule 'load' operation ('values\_load\_2', smvmProject/smvm.cpp:34) on array 'values' due to limited memory ports. Please consider using a memory core with more ports or partitioning the array 'values'.}

\begin{table}[H]
	\centering
	\begin{tabular}{|c|c|c|c|c|c|c|c|c|}
		\hline
		\multicolumn{1}{|c|}{Loop} & \multicolumn{2}{|c|}{\textbf{Latency}} & \multicolumn{1}{c|}{\textbf{Iteration Latency}} & \multicolumn{2}{c|}{\textbf{Initiation Interval}} & \multicolumn{1}{c|}{\textbf{Trip Count}}  \\
		Name & min & max &  & achieved & target &  \\
		\hline
		- loop1 & 32 & 40 & 8$\sim$10 & - & - & 4 \\
		+ loop2 & 4 & 6 & 5 & \textcolor{red}{2} & 1 & 0$\sim$1 \\
		\hline
	\end{tabular}
	\caption{HLS Solution 7 with columnIndex partitioning Latency Loops Summary}
	\label{tab:hls-solution-7-columnindex-partitioning-loop-summary}
\end{table}

Si può notare come in questo caso il warning sia relativo all'array \textit{values}. In particolare, la tipologia di warning è la medesima facendo presupporre che il tool non riesca a schedulare correttamente, secondo le direttive imposte dall'architettura, gli accessi in parallello all'array \textit{values}. Infatti, il valore di Initiation Interval raggiunto risulta essere ancora maggiore di quello di quello target. 
\\
Pertanto, si potrebbe aggiungere una direttiva di partizionamento relativa all'array menzionato all'interno del log, cioè \textit{values}.

\lstinputlisting[language=C++]{solutions/s7/s7values.cpp}

Effettuando nuovamente la sintesi, si ottiene il seguente log nella console e i seguenti valori di latenza.
\\
\textcolor{blue}{WARNING: [SCHED 204-69] Unable to schedule 'load' operation ('x\_load\_2', smvmProject/smvm.cpp:34) on array 'x' due to limited memory ports. Please consider using a memory core with more ports or partitioning the array 'x'.}

\begin{table}[H]
	\centering
	\begin{tabular}{|c|c|c|c|c|c|c|c|c|}
		\hline
		\multicolumn{1}{|c|}{Loop} & \multicolumn{2}{|c|}{\textbf{Latency}} & \multicolumn{1}{c|}{\textbf{Iteration Latency}} & \multicolumn{2}{c|}{\textbf{Initiation Interval}} & \multicolumn{1}{c|}{\textbf{Trip Count}}  \\
		Name & min & max &  & achieved & target &  \\
		\hline
		- loop1 & 32 & 40 & 8$\sim$10 & - & - & 4 \\
		+ loop2 & 4 & 6 & 5 & \textcolor{red}{2} & 1 & 0$\sim$1 \\
		\hline
	\end{tabular}
	\caption{HLS Solution 7 with columnIndex and values partitioning Latency Loops Summary}
	\label{tab:hls-solution-7-columnindex-values-partitioning-loop-summary}
\end{table}

Si può notare come in questo caso il warning sia relativo all'array \textit{x}. In particolare, la tipologia di warning è la medesima della precedente. Anche in questo caso il valore di Initiation Interval raggiunto risulta essere ancora maggiore di quello di quello target. 
\\
Pertanto, si potrebbe aggiungere una direttiva di partizionamento relativa all'array menzionato all'interno del log, cioè \textit{x}.

\lstinputlisting[language=C++]{solutions/s7/s7x.cpp}

Effettuando nuovamente la sintesi, si ottiene il seguente log nella console e il seguente report.

\textcolor{blue}{WARNING: [SCHED 204-21] Estimated clock period (10.208ns) exceeds the target (target clock period: 10ns, clock uncertainty: 1.25ns, effective delay budget: 8.75ns).}
\textcolor{blue}{WARNING: [SCHED 204-21] The critical path in module 'smvm' consists of the following:}
\\
\textcolor{blue}{'add' operation ('ytmp\_1\_3', smvmProject/smvm.cpp:34) [139]  (2.55 ns)}
\\
\textcolor{blue}{'phi' operation ('ytmp', smvmProject/smvm.cpp:34) with incoming values : ('ytmp\_1\_3', smvmProject/smvm.cpp:34) [68]  (0 ns)}
\\
\textcolor{blue}{'add' operation ('ytmp\_1', smvmProject/smvm.cpp:34) [105]  (2.55 ns)}
\\
\textcolor{blue}{'add' operation ('ytmp\_1\_1', smvmProject/smvm.cpp:34) [117]  (2.55 ns)}
\\
\textcolor{blue}{'add' operation ('ytmp\_1\_2', smvmProject/smvm.cpp:34) [128]  (2.55 ns)}

\begin{table}[H]
	\centering
	\begin{minipage}[t]{0.45\linewidth}
		\centering
		\begin{tabular}{|c|c|c|c|}
			\hline
			\textbf{Clock} & \textbf{Target} & \textbf{Estimated} & \textbf{Uncertainty} \\
			\hline
			ap\_clk & 10.00 & 10.208 & 1.25 \\
			\hline
		\end{tabular}
		\caption{HLS Solution 7 with columnIndex, values and x partitioning Timing Summary (ns)}
		\label{tab:hls-solution-7-columnindex-values-x-partitioning-timing-summary}
	\end{minipage}
	\hfill
	\begin{minipage}[t]{0.45\linewidth}
		\centering
		\begin{tabular}{|c|c|c|c|}
			\hline
			\multicolumn{2}{|c|}{\textbf{Latency}} & \multicolumn{2}{|c|}{\textbf{Interval}} \\
			min & max & min & max \\
			\hline
			29 & 33 & 29 & 33 \\
			\hline
		\end{tabular}
		\caption{HLS Solution 7 with columnIndex, values and x partitioning Latency Summary (clock cycles)}
		\label{tab:hls-solution-7-columnindex-values-x-partitioning-latency-summary}
	\end{minipage}
\end{table}

\begin{table}[H]
	\centering
	\begin{tabular}{|c|c|c|c|c|c|c|c|c|}
		\hline
		\multicolumn{1}{|c|}{Loop} & \multicolumn{2}{|c|}{\textbf{Latency}} & \multicolumn{1}{c|}{\textbf{Iteration Latency}} & \multicolumn{2}{c|}{\textbf{Initiation Interval}} & \multicolumn{1}{c|}{\textbf{Trip Count}}  \\
		Name & min & max &  & achieved & target &  \\
		\hline
		- loop1 & 28 & 32 & 7$\sim$8 & - & - & 4 \\
		+ loop2 & 3 & 4 & 4 & 1 & 1 & 0$\sim$1 \\
		\hline
	\end{tabular}
	\caption{HLS Solution 7 with columnIndex, values and x partitioning Latency Loops Summary}
	\label{tab:hls-solution-7-columnindex-values-x-partitioning-loop-summary}
\end{table}

Si può evidenziare come, in questo caso, il valore di Initiation Interval raggiunto sia il medesimo di quello target, cioè pari a 1. Inoltre, quello che si può notare è che all'interno della console viene visualizzato un warning indicante un periodo di clock stimato maggiore di quello target. In particolare, viene stimato un timing per ogni operazione in maniera dettagliata: $2.55 ns$ per \textit{add operation ytmp\_1\_3}, $0 ns$ per \textit{phi operation ytmp}, $2.55 ns$ per \textit{add operation ytmp\_1}, $2.55 ns$ per \textit{add operation ytmp\_1\_1} e $2.55 ns$ per \textit{add operation ytmp\_1\_2}. Pertanto, calcolando la somma di tutti questi timing stimati si ottiene un periodo di clock stimato pari a $10.2 ns$. Nello specifico, il periodo di clock rimanente, cioè $0.008 ns$, evidentemente corrisponde al valore di timing relativo a \textit{phi operation ytmp} che viene approssimato all'interno del report a $0 ns$. 
\\
Inoltre, si può notare come l'utilizzazione delle risorse sia notevolmente aumentata. In particolare, l'utilizzazione delle risorse, rispetto alle risorse disponibili della scheda, risultano essere pari al $5\%$ per i DSP e al $2\%$ per le LUT.

\begin{table}[H]
	\centering
	\begin{tabular}{|l|c|c|c|c|}
		\hline
		\textbf{Name}    & \textbf{BRAM\_18K} & \textbf{DSP48E} & \textbf{FF} & \textbf{LUT} \\ \hline
		DSP              & -                   & -               & -           & -            \\ 
		Expression       & -                   & 12              & 0           & 534          \\ 
		FIFO             & -                   & -               & -           & -            \\ 
		Instance         & -                   & -               & -           & 252            \\ 
		Memory           & 0                   & -               & -          & -            \\ 
		Multiplexer      & -                   & -               & -           & 99          \\ 
		Register         & -                   & -               & 854         & 128            \\ \hline
		\textbf{Total}   & 0                   & 12               & 854         & 1013          \\ \hline
		\textbf{Available} & 280               & 220             & 106400      & 53200        \\ \hline
		\textbf{Utilization (\%)} & 0            & 5               & $\sim$0     & 1      \\ \hline
	\end{tabular}
	\caption{HLS Solution 7 with columnIndex, values and x partitioning Utilization Estimates Summary}
	\label{tab:hls-solution-7-columnindex-values-x-partitioning-utilization-estimates-summary}
\end{table}

Si procede con successivi passi così da verificare se tale problematica, riguardo il periodo di clock stimato superiore a quello target, possa essere risolta dal tool, tramite ulteriori ottimizzazioni, durante la fase di Export RTL.

\begin{table}[H]
	\centering
	\begin{tabular}{|c|c|c|c|c|c|c|c|}
		\hline
		\multicolumn{1}{|c|}{RTL} & \multicolumn{1}{|c|}{Status} & \multicolumn{3}{c|}{\textbf{Latency}} & \multicolumn{3}{c|}{\textbf{Interval}} \\
		&  & min & avg & max & min & avg & max \\
		\hline
		VHDL & Pass & 29 & 29 & 29 & NA & NA & NA \\
		\hline
	\end{tabular}
	\caption{HLS Solution 7 with columnIndex, values and x partitioning C/RTL Cosimulation Summary }
	\label{tab:hls-solution-7-columnindex-values-x-partitioning-cosimulation-summary}
\end{table}

Si può notare come il tool sia riuscito a risolvere la problematica riguardante il periodo di clock stimato superiore a quello target. Infatti, si evidenzia come quello raggiunto post-implementation risulta essere pari a $8.110 ns$. Bisogna notare, però, che l'utilizzazione delle risorse risulta essere notevolmente alta dal momento che il tool è riuscito ad attuare i partizionamenti di fattore pari a 4 su tutti e tre gli array precedentemente citati.

\begin{table}[H]
	\centering
	\begin{minipage}[t]{0.45\linewidth}
		\centering
		\begin{tabular}{|l|r|}
			\hline
			\textbf{Resource} & \textbf{VHDL} \\
			\hline
			SLICE & 314 \\
			\hline
			LUT & 915 \\
			\hline
			FF & 297 \\
			\hline
			DSP & 12 \\
			\hline
			BRAM & 0 \\
			\hline
			SRL & 0 \\
			\hline
		\end{tabular}
		\caption{HLS Solution 7 with columnIndex, values and x partitioning Export RTL Resource Usage}
		\label{tab:hls-solution-7-columnindex-values-x-partitioning-export-rtl-resoruce-usage}
	\end{minipage}
	\hfill
	\begin{minipage}[t]{0.45\linewidth}
		\centering
		\begin{tabular}{|l|r|}
			\hline
			\textbf{Timing} & \textbf{VHDL} \\
			\hline
			CP required & 10.000 \\
			\hline
			CP achieved post-synthesis & 7.449 \\
			\hline
			CP achieved post-implementation & 8.110 \\
			\hline
		\end{tabular}
		\caption{HLS Solution 7 with columnIndex, values and x partitioning Export RTL Final Timing}
		\label{tab:hls-solution-7-columnindex-values-x-partitioning-export-rtl-final-timing}
	\end{minipage}
\end{table}
