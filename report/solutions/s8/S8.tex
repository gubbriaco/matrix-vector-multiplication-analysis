Nella soluzione hardware in questione verrà utilizzata la direttiva di pipeline, la direttiva di unrolling di fattore pari a 2 e la direttiva di partizionamento di tipologia cyclic. Nello specifico, verranno analizzate le seguenti implementazioni relative al loop2:
\begin{itemize}
	\item Pipeline, Unroll=2, Cyclic=4 (columnIndex, values, x)
	\item Pipeline, Unroll=2, Cyclic=4 (columnIndex)
	\item Pipeline, Unroll=2, Cyclic=4 (values)
	\item Pipeline, Unroll=2, Cyclic=4 (x)
\end{itemize}

In particolare, è possibile evidenziare nel dettaglio le differenti soluzioni hardware nei seguenti allegati.
\lstinputlisting[language=C++]{solutions/s8/s8all3.cpp}
\lstinputlisting[language=C++]{solutions/s8/s8columnIndex.cpp}
\lstinputlisting[language=C++]{solutions/s8/s8values.cpp}
\lstinputlisting[language=C++]{solutions/s8/s8x.cpp}

Effettuando la sintesi è possibile evidenziare il seguente report:\\

\begin{table}[H]
	\centering
	\begin{tabular}{|c|c|c|c|c|}
		\hline
		\textbf{Solution} & \textbf{Clock} & \textbf{Target} & \textbf{Estimated} & \textbf{Uncertainty} \\
		\hline
		columnIndex, values, x & ap\_clk & 10.00 & 8.510 & 1.25 \\
		\hline
		columnIndex & ap\_clk & 10.00 & 8.510 & 1.25 \\
		\hline
		values & ap\_clk & 10.00 & 8.510 & 1.25 \\
		\hline
		x & ap\_clk & 10.00 & 8.510 & 1.25 \\
		\hline
	\end{tabular}
	\caption{HLS Solution 8 Timing Summary (ns)}
	\label{tab:hls-solution-8-timing-summary}
\end{table}

\begin{table}[H]
	\centering
	\begin{tabular}{|c|c|c|c|c|}
		\hline
		\multicolumn{1}{|c|}{\textbf{Solution}} & \multicolumn{2}{|c|}{\textbf{Latency}} & \multicolumn{2}{|c|}{\textbf{Interval}} \\
		& min & max & min & max \\
		\hline
		columnIndex, values, x & 29 & 37 & 29 & 37 \\
		\hline
		columnIndex & 33 & 41 & 33 & 41 \\
		\hline
		values & 33 & 41 & 33 & 41 \\
		\hline
		x & 29 & 37 & 29 & 37 \\
		\hline
	\end{tabular}
	\caption{HLS Solution 8 Latency Summary (clock cycles)}
	\label{tab:hls-solution-8-latency-summary}
\end{table}

Si può notare come i valori di latenza totale, delle latenze totali dei loop corrispondenti e di Iteration Latency risultano essere differenti tra le varia solution. In particolare, la soluzione che implementa il partizionamento dei tre array (columnIndex, values e x) presenta valori di latenza uguali a quelli della soluzione che implementa il partizionamento dell'array x. Bisogna notare, però, che il valore del trip count risulta essere il medesimo per ogni solution. In particolare, essendo previsto un unrolling di fattore pari a 2 all'interno del loop2, il corrispondente valore risulta essere dimezzato rispetto alla solution 1 in cui non è presente alcun pragma di parallelismo. 

\begin{table}[H]
	\centering
	\begin{tabular}{|c|c|c|c|c|c|c|c|c|c|}
		\hline
		\multicolumn{1}{|c|}{\textbf{Solution}} & \multicolumn{1}{|c|}{Loop Name} & \multicolumn{2}{|c|}{\textbf{Latency}} & \multicolumn{1}{c|}{\textbf{Iteration Latency}} & \multicolumn{2}{c|}{\textbf{Initiation Interval}} & \multicolumn{1}{c|}{\textbf{Trip}}  \\
		&  & min & max & & achieved & target & \textbf{Count} \\
		\hline
		columnIndex, values, x & - loop1 & 28 & 36 & 7$\sim$9 & - & - & 4 \\
		& + loop2 & 3 & 5 & 4 & 1 & 1 & 0$\sim$2 \\
		\hline
		columnIndex & - loop1 & 32 & 40 & 8$\sim$10 & - & - & 4 \\
		& + loop2 & 4 & 6 & 5 & 1 & 1 & 0$\sim$2 \\
		\hline
		values & - loop1 & 32 & 40 & 8$\sim$10 & - & - & 4 \\
		& + loop2 & 4 & 6 & 5 & 1 & 1 & 0$\sim$2 \\
		\hline
		x & - loop1 & 28 & 36 & 7$\sim$9 & - & - & 4 \\
		& + loop2 & 3 & 5 & 4 & 1 & 1 & 0$\sim$2 \\
		\hline
	\end{tabular}
	\caption{HLS Solution 8 Latency Loops Summary }
	\label{tab:hls-solution-8-loop-summary}
\end{table}

\begin{table}[H]
	\centering
	\begin{tabular}{|c|c|c|c|c|}
		\hline
		\textbf{Solution} & \textbf{BRAM\_18K} & \textbf{DSP48E} & \textbf{FF} & \textbf{LUT} \\
		\hline
		columnIndex, values, x & 0 & 6 & 630 & 628 \\
		\hline
		columnIndex & 0 & 6 & 535 & 560 \\
		\hline
		values & 0 & 6 & 634 & 581 \\
		\hline
		x & 0 & 6 & 596 & 441 \\
		\hline
	\end{tabular}
	\caption{HLS Solution 8 Utilization Estimates [\#]}
	\label{tab:hls-solution-8-utilization-report}
\end{table}

\begin{table}[H]
	\centering
	\begin{tabular}{|c|c|c|c|c|c|c|c|c|}
		\hline
		\multicolumn{1}{|c|}{\textbf{Solution}} & \multicolumn{1}{|c|}{RTL} & \multicolumn{1}{|c|}{Status} & \multicolumn{3}{c|}{\textbf{Latency}} & \multicolumn{3}{c|}{\textbf{Interval}} \\
		& &  & min & avg & max & min & avg & max \\
		\hline
		columnIndex, values, x & VHDL & Pass & 33 & 33 & 33 & NA & NA & NA \\
		\hline
		columnIndex & VHDL & Pass & 37 & 37 & 37 & NA & NA & NA \\
		\hline
		values & VHDL & Pass & 37 & 37 & 37 & NA & NA & NA \\
		\hline
		x & VHDL & Pass & 33 & 33 & 33 & NA & NA & NA \\
		\hline
	\end{tabular}
	\caption{HLS Solution 8 C/RTL Cosimulation Report }
	\label{tab:hls-solution-8-cosimulation-report}
\end{table}

Si può notare come, in corrrispondenza della soluzione hardware basata sul partizionamento dei tre array, si ha la maggiore utilizzazione di slice, LUT e FF. Inoltre, si può evidenziare come le soluzioni basate rispettivamente sul partitioning di columnIndex e values risultano avere pressoché la medesima utilizzazione dal momento che le dimensioni di tali strutture dati è la medesima. In particolare, rispetto alla soluzione hardware similare, dove era stato utilizzato un fattore di partitioning pari a 2 di tipologia cyclic (solution 6), si può notare, in corrispondenza del partizionamento dei tre array, un incremento dei FF di circa il $59\%$, un incremento delle LUT e delle slice di circa il $9\%$.

\begin{table}[H]
	\centering
	\begin{tabular}{|c|c|c|c|c|c|c|c|c|}
		\hline
		\textbf{Solution} & \textbf{SLICE} & \textbf{LUT} & \textbf{FF} & \textbf{DSP} & \textbf{BRAM} & \textbf{CP} & \textbf{CP} & \textbf{CP} \\
		& & & & & & \textbf{required} & \textbf{achieved} & \textbf{achieved}\\
		& & & & & & & \textbf{post-} & \textbf{post-}\\
		& & & & & & & \textbf{synthesis} & \textbf{implementation}\\
		\hline
		columnIndex, values, x  & 123 & 343 & 315 & 6 & 0 & 10 & 6.540 & 6.571 \\
		\hline
		columnIndex  & 85 & 261 & 226 & 6 & 0 & 10 & 7.496 & 7.654 \\
		\hline
		values  & 92 & 274 & 196 & 6 & 0 & 10 & 7.927 & 7.780 \\
		\hline
		x  & 104 & 248 & 313 & 6 & 0 & 10 & 6.540 & 6.844 \\
		\hline
	\end{tabular}
	\caption{HLS Solution 8 Export RTL Report}
	\label{tab:hls-solution-8-export-rtl-report}
\end{table}
