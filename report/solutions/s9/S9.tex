Qui, di seguito, viene riportata l'architettura relativa alla nona solution.

\lstinputlisting[language=C++]{solutions/s9/s9.cpp}

In particolare, rispetto alla soluzione hardware 5 e 7 dove rispettivamente era stato considerato un parallelismo di fattore pari a 2 e un parallelismo di fattore pari a 4, in questa solution è stato considerato un unrolling di fattore pari a 8. In particolare, ciò che ci si aspetta è un aumento delle risorse ed eventuali problematiche relative al timing, similari a quelle riscontrate nella solution 7, dal momento che il tool deve gestire all'interno del loop2 più accessi in memoria paralleli.
\\
Effettuando la sintesi è possibile evidenziare il seguente log nella console:
\\
\textcolor{blue}{WARNING: [SCHED 204-69] Unable to schedule 'load' operation ('columnIndex\_load\_6', smvmProject/smvm.cpp:34) on array 'columnIndex' due to limited memory ports. Please consider using a memory core with more ports or partitioning the array 'columnIndex'.}
\\
Tale log sta a significare che non riesce a schedulare correttamente, dal punto di vista degli accessi in memoria, la load operation relativa all'array \textit{columnIndex} dato dal numero limitato di porte relative alla memoria.
\\
In particolare, analizzando il report relativo alla sintesi, si può notare come l'Initiation Interval, associato al loop2, raggiunto risulta essere maggiore di quello target.
\\

\begin{table}[H]
	\centering
	\begin{minipage}[t]{0.45\linewidth}
		\centering
		\begin{tabular}{|c|c|c|c|}
			\hline
			\textbf{Clock} & \textbf{Target} & \textbf{Estimated} & \textbf{Uncertainty} \\
			\hline
			ap\_clk & 10.00 & 8.510 & 1.25 \\
			\hline
		\end{tabular}
		\caption{HLS Solution 9 Timing Summary (ns)}
		\label{tab:hls-solution-9-timing-summary}
	\end{minipage}
	\hfill
	\begin{minipage}[t]{0.45\linewidth}
		\centering
		\begin{tabular}{|c|c|c|c|}
			\hline
			\multicolumn{2}{|c|}{\textbf{Latency}} & \multicolumn{2}{|c|}{\textbf{Interval}} \\
			min & max & min & max \\
			\hline
			45 & 61 & 45 & 61 \\
			\hline
		\end{tabular}
		\caption{HLS Solution 9 Latency Summary (clock cycles)}
		\label{tab:hls-solution-9-latency-summary}
	\end{minipage}
\end{table}

\begin{table}[H]
	\centering
	\begin{tabular}{|c|c|c|c|c|c|c|c|c|}
		\hline
		\multicolumn{1}{|c|}{Loop} & \multicolumn{2}{|c|}{\textbf{Latency}} & \multicolumn{1}{c|}{\textbf{Iteration Latency}} & \multicolumn{2}{c|}{\textbf{Initiation Interval}} & \multicolumn{1}{c|}{\textbf{Trip Count}}  \\
		Name & min & max &  & achieved & target &  \\
		\hline
		- loop1 & 44 & 60 & 11$\sim$15 & - & - & 4 \\
		+ loop2 & 7 & 11 & 8 & \textcolor{red}{4} & 1 & 0$\sim$1 \\
		\hline
	\end{tabular}
	\caption{HLS Solution 9 Latency Loops Summary}
	\label{tab:hls-solution-9-loop-summary}
\end{table}

Pertanto, si potrebbe aggiungere una direttiva di partizionamento con fattore pari a 8 relativa all'array menzionato all'interno del log, cioè \textit{columnIndex}.

\lstinputlisting[language=C++]{solutions/s9/s9columnIndex.cpp}

Effettuando nuovamente la sintesi, si ottiene il seguente log nella console e i seguenti valori di latenza.
\\
\textcolor{blue}{WARNING: [SCHED 204-69] Unable to schedule 'load' operation ('values\_load\_6', smvmProject/smvm.cpp:34) on array 'values' due to limited memory ports. Please consider using a memory core with more ports or partitioning the array 'values'.}

\begin{table}[H]
	\centering
	\begin{tabular}{|c|c|c|c|c|c|c|c|c|}
		\hline
		\multicolumn{1}{|c|}{Loop} & \multicolumn{2}{|c|}{\textbf{Latency}} & \multicolumn{1}{c|}{\textbf{Iteration Latency}} & \multicolumn{2}{c|}{\textbf{Initiation Interval}} & \multicolumn{1}{c|}{\textbf{Trip Count}}  \\
		Name & min & max &  & achieved & target &  \\
		\hline
		- loop1 & 44 & 60 & 11$\sim$15 & - & - & 4 \\
		+ loop2 & 7 & 11 & 8 & \textcolor{red}{4} & 1 & 0$\sim$1 \\
		\hline
	\end{tabular}
	\caption{HLS Solution 9 with columnIndex partitioning Latency Loops Summary}
	\label{tab:hls-solution-9-columnindex-partitioning-loop-summary}
\end{table}

Si può notare come in questo caso il warning sia relativo all'array \textit{values}. In particolare, la tipologia di warning è la medesima facendo presupporre che il tool non riesca a schedulare correttamente, secondo le direttive imposte dall'architetture, gli accessi in parallello all'array \textit{values}. Infatti, il valore di Iteration Latency raggiunto risulta essere ancora maggiore di quello di quello target. 
\\
Pertanto, si potrebbe aggiungere una direttiva di partizionamento relativa all'array menzionato all'interno del log, cioè \textit{values}.

\lstinputlisting[language=C++]{solutions/s9/s9values.cpp}

Effettuando nuovamente la sintesi, si ottiene il seguente log nella console e i seguenti valori di latenza.
\\
\textcolor{blue}{WARNING: [SCHED 204-69] Unable to schedule 'load' operation ('x\_load\_6', smvmProject/smvm.cpp:34) on array 'x' due to limited memory ports. Please consider using a memory core with more ports or partitioning the array 'x'.}

\begin{table}[H]
	\centering
	\begin{tabular}{|c|c|c|c|c|c|c|c|c|}
		\hline
		\multicolumn{1}{|c|}{Loop} & \multicolumn{2}{|c|}{\textbf{Latency}} & \multicolumn{1}{c|}{\textbf{Iteration Latency}} & \multicolumn{2}{c|}{\textbf{Initiation Interval}} & \multicolumn{1}{c|}{\textbf{Trip Count}}  \\
		Name & min & max &  & achieved & target &  \\
		\hline
		- loop1 & 44 & 60 & 11$\sim$15 & - & - & 4 \\
		+ loop2 & 7 & 11 & 8 & \textcolor{red}{4} & 1 & 0$\sim$1 \\
		\hline
	\end{tabular}
	\caption{HLS Solution 9 with columnIndex and values partitioning Latency Loops Summary}
	\label{tab:hls-solution-9-columnindex-values-partitioning-loop-summary}
\end{table}

Si può notare come in questo caso il warning sia relativo all'array \textit{x}. In particolare, la tipologia di warning è la medesima della precedente. Anche in questo caso il valore di Iteration Latency raggiunto risulta essere ancora maggiore di quello di quello target. 
\\
Pertanto, si potrebbe aggiungere una direttiva di partizionamento relativa all'array menzionato all'interno del log, cioè \textit{x}.

\lstinputlisting[language=C++]{solutions/s9/s9x.cpp}

Effettuando nuovamente la sintesi, si ottiene il seguente log nella console e il seguente report.
\\
\textcolor{blue}{WARNING: [SCHED 204-68] The II Violation in module 'smvm': Unable to enforce a carried dependence constraint (II = 1, distance = 1, offset = 0)
	between 'add' operation ('ytmp\_1\_7', smvmProject/smvm.cpp:34) and 'add' operation ('ytmp\_1', smvmProject/smvm.cpp:34).}
\\
\textcolor{blue}{WARNING: [SCHED 204-68] The II Violation in module 'smvm': Unable to enforce a carried dependence constraint (II = 1, distance = 1, offset = 1)
	between 'add' operation ('ytmp\_1\_5', smvmProject/smvm.cpp:34) and 'add' operation ('ytmp\_1\_2', smvmProject/smvm.cpp:34).}
\\
\textcolor{blue}{WARNING: [SCHED 204-68] The II Violation in module 'smvm': Unable to enforce a carried dependence constraint (II = 2, distance = 1, offset = 1)
	between 'add' operation ('ytmp\_1\_6', smvmProject/smvm.cpp:34) and 'add' operation ('ytmp\_1\_2', smvmProject/smvm.cpp:34).}
\\

In particolare, tali warning suggeriscono problematiche relative alla variabile ytmp. Nello specifico, ymtp è una variabile temporanea (di appoggio) di tipo DTYPE, cioè int, e pertanto il partizionamento su tale variabile non avrebbe senso poiché quest'ultimo funziona solo con gli array. Bisogna notare però che il problema non è relativo a ytmp ma è associato a y, cioè l'output, che è un vettore. Inoltre, y fa riferimento al loop1 e questo suggerisce che gli eventuali problemi sono relativi al ciclo1 ed, essendo che il loop1 presenta al suo interno il loop2, allora tali problemi si ripercuotono sul loop2. In aggiunta, si evidenzia come in questo caso l'Initiation Interval raggiunto risulta essere decrementato da 4 a 3 ma comunque ancora maggiore di quello target. 

\begin{table}[H]
	\centering
	\begin{tabular}{|c|c|c|c|c|c|c|c|c|}
		\hline
		\multicolumn{1}{|c|}{Loop} & \multicolumn{2}{|c|}{\textbf{Latency}} & \multicolumn{1}{c|}{\textbf{Iteration Latency}} & \multicolumn{2}{c|}{\textbf{Initiation Interval}} & \multicolumn{1}{c|}{\textbf{Trip Count}}  \\
		Name & min & max &  & achieved & target &  \\
		\hline
		- loop1 & 40 & 52 & 10$\sim$13 & - & - & 4 \\
		+ loop2 & 6 & 9 & 7 & \textcolor{red}{3} & 1 & 0$\sim$1 \\
		\hline
	\end{tabular}
	\caption{HLS Solution 9 with columnIndex, values and x partitioning Latency Loops Summary}
	\label{tab:hls-solution-9-columnindex-values-partitioning-loop-summary}
\end{table}

Un'idea potrebbe quella di effettuare un partizionamento anche sull'array rowPtr.

\lstinputlisting[language=C++]{solutions/s9/s9rowPtr4.cpp}

Effettuando nuovamente la sintesi, si ottiene il seguente log nella console e il seguente report.
\\
\textcolor{blue}{WARNING: [SCHED 204-68] The II Violation in module 'smvm': Unable to enforce a carried dependence constraint (II = 1, distance = 1, offset = 0)
	between 'add' operation ('ytmp\_1\_7', smvmProject/smvm.cpp:34) and 'add' operation ('ytmp\_1', smvmProject/smvm.cpp:34).}
\\
\textcolor{blue}{WARNING: [SCHED 204-68] The II Violation in module 'smvm': Unable to enforce a carried dependence constraint (II = 1, distance = 1, offset = 1)
	between 'add' operation ('ytmp\_1\_5', smvmProject/smvm.cpp:34) and 'add' operation ('ytmp\_1\_2', smvmProject/smvm.cpp:34).}
\\
\textcolor{blue}{WARNING: [SCHED 204-68] The II Violation in module 'smvm': Unable to enforce a carried dependence constraint (II = 2, distance = 1, offset = 1)
	between 'add' operation ('ytmp\_1\_6', smvmProject/smvm.cpp:34) and 'add' operation ('ytmp\_1\_2', smvmProject/smvm.cpp:34).}
\\
In particolare, si riscontrano i medesimi warning allegati precedentemente.

\begin{table}[H]
	\centering
	\begin{tabular}{|c|c|c|c|c|c|c|c|c|}
		\hline
		\multicolumn{1}{|c|}{Loop} & \multicolumn{2}{|c|}{\textbf{Latency}} & \multicolumn{1}{c|}{\textbf{Iteration Latency}} & \multicolumn{2}{c|}{\textbf{Initiation Interval}} & \multicolumn{1}{c|}{\textbf{Trip Count}}  \\
		Name & min & max &  & achieved & target &  \\
		\hline
		- loop1 & 44 & 56 & 11$\sim$14 & - & - & 4 \\
		+ loop2 & 6 & 9 & 7 & \textcolor{red}{3} & 1 & 0$\sim$1 \\
		\hline
	\end{tabular}
	\caption{HLS Solution 9 with columnIndex, values, x and rowPtr partitioning Latency Loops Summary}
	\label{tab:hls-solution-9-columnindex-values-partitioning-loop-summary}
\end{table}

Pertanto, a tale proposito si potrebbe effettuare un partizionamento anche su y.

\lstinputlisting[language=C++]{solutions/s9/s9y4.cpp}

Effettuando nuovamente la sintesi, si ottiene il seguente log nella console e il seguente report.
\\
\textcolor{blue}{WARNING: [SCHED 204-68] The II Violation in module 'smvm': Unable to enforce a carried dependence constraint (II = 1, distance = 1, offset = 0)
	between 'add' operation ('ytmp\_1\_7', smvmProject/smvm.cpp:35) and 'add' operation ('ytmp\_1', smvmProject/smvm.cpp:35).}
\\
\textcolor{blue}{WARNING: [SCHED 204-68] The II Violation in module 'smvm': Unable to enforce a carried dependence constraint (II = 1, distance = 1, offset = 1)
	between 'add' operation ('ytmp\_1\_5', smvmProject/smvm.cpp:35) and 'add' operation ('ytmp\_1\_2', smvmProject/smvm.cpp:35).}
\\
\textcolor{blue}{WARNING: [SCHED 204-68] The II Violation in module 'smvm': Unable to enforce a carried dependence constraint (II = 2, distance = 1, offset = 1)
	between 'add' operation ('ytmp\_1\_6', smvmProject/smvm.cpp:35) and 'add' operation ('ytmp\_1\_2', smvmProject/smvm.cpp:35).}
\\
Nello specifico, si riscontrano nuovamente i medesimi warning allegati precedentemente.

\begin{table}[H]
	\centering
	\begin{tabular}{|c|c|c|c|c|c|c|c|c|}
		\hline
		\multicolumn{1}{|c|}{Loop} & \multicolumn{2}{|c|}{\textbf{Latency}} & \multicolumn{1}{c|}{\textbf{Iteration Latency}} & \multicolumn{2}{c|}{\textbf{Initiation Interval}} & \multicolumn{1}{c|}{\textbf{Trip Count}}  \\
		Name & min & max &  & achieved & target &  \\
		\hline
		- loop1 & 44 & 56 & 11$\sim$14 & - & - & 4 \\
		+ loop2 & 6 & 9 & 7 & \textcolor{red}{3} & 1 & 0$\sim$1 \\
		\hline
	\end{tabular}
	\caption{HLS Solution 9 with columnIndex, values, x, rowPtr and y partitioning Latency Loops Summary}
	\label{tab:hls-solution-9-columnindex-values-partitioning-loop-summary}
\end{table}

Pertanto, dal momento che i warning sono ancora presenti e il valore di Initiation Latency non tende a diminuire, si potrebbe provare effettuando anche il pipelining del loop1.

\lstinputlisting[language=C++]{solutions/s9/s9pipelineloop1.cpp}

Effettuando nuovamente la sintesi, si ottiene il seguente log nella console e il seguente report.
\\
\textcolor{blue}{WARNING: [XFORM 203-503] Ignored partial unroll directive for loop 'loop2' (smvmProject/smvm.cpp:21) because its parent loop or function is pipelined.}
\\
\textcolor{blue}{WARNING: [XFORM 203-503] Cannot unroll loop 'loop2' (smvmProject/smvm.cpp:21) in function 'smvm' completely: variable loop bound.}
\\
\textcolor{blue}{WARNING: [SCHED 204-65] Unable to satisfy pipeline directive: Loop contains subloop(s) not being unrolled or flattened.}
\\
\begin{table}[H]
	\centering
	\begin{tabular}{|c|c|c|c|c|c|c|c|c|}
		\hline
		\multicolumn{1}{|c|}{Loop} & \multicolumn{2}{|c|}{\textbf{Latency}} & \multicolumn{1}{c|}{\textbf{Iteration Latency}} & \multicolumn{2}{c|}{\textbf{Initiation Interval}} & \multicolumn{1}{c|}{\textbf{Trip Count}}  \\
		Name & min & max &  & achieved & target &  \\
		\hline
		- loop1 & 16 & 40 & 4$\sim$10 & - & - & 4 \\
		+ loop2 & 0 & 6 & 4 & 1 & 1 & 0$\sim$4 \\
		\hline
	\end{tabular}
	\caption{HLS Solution 9 with columnIndex, values and x partitioning and loop1 pipelined Latency Loops Summary}
	\label{tab:hls-solution-9-columnindex-values-partitioning-loop-summary}
\end{table}

Si può notare come i warning relativi a ytmp non sono più presenti all'interno della console e che, inoltre, l'Initiation Interval relativo al loop2 presenta un valore raggiunto uguale a quello target, cioè pari a 1. In aggiunta, si possono evidenziare ulteriori warning all'interno della console. Essi sono i medesimi di quelli riscontrati nella solution 7, in cui veniva effettuato un unrolling di fattore pari a 4 nel loop2, dopo aver applicato il pipeline al loop1. Anche in questo caso, il tool non è riuscito a soddisfare la richiesta di pipeline a causa dei bound non noti e, pertanto, non riuscendo nemmeno a soddisfare la direttiva di unrolling all'interno del loop2. Infatti, si può notare come i valori di Initiation Interval (achieved e target) risultano essere non definiti, mentre il trip count relativo relativo al loop2 risulta essere il medesimo di quello della solution2 dove non era presente alcun pragma di parallelismo nel ciclo 2.
\\
Pertanto, considerando l'implementazione finale ottenuta in corrispondenza della solution in questione, si effettua la C/RTL Cosimulation e l'Export RTL e si analizzano, di conseguenza, i report corrispondenti.

\begin{table}[H]
	\centering
	\begin{tabular}{|c|c|c|c|c|c|c|c|}
		\hline
		\multicolumn{1}{|c|}{RTL} & \multicolumn{1}{|c|}{Status} & \multicolumn{3}{c|}{\textbf{Latency}} & \multicolumn{3}{c|}{\textbf{Interval}} \\
		&  & min & avg & max & min & avg & max \\
		\hline
		VHDL & Pass & 34 & 34 & 34 & NA & NA & NA \\
		\hline
	\end{tabular}
	\caption{HLS Solution 9 with columnIndex, values and x partitioning and loop1 pipelined C/RTL Cosimulation Summary }
	\label{tab:hls-solution-7-loop1-pipeline-cosimulation-summary}
\end{table}

\begin{table}[H]
	\centering
	\begin{minipage}[t]{0.45\linewidth}
		\centering
		\begin{tabular}{|l|r|}
			\hline
			\textbf{Resource} & \textbf{VHDL} \\
			\hline
			SLICE & 80 \\
			\hline
			LUT & 208 \\
			\hline
			FF & 263 \\
			\hline
			DSP & 3 \\
			\hline
			BRAM & 0 \\
			\hline
			SRL & 0 \\
			\hline
		\end{tabular}
		\caption{HLS Solution 9 with columnIndex, values and x partitioning and loop1 pipelined Export RTL Resource Usage}
		\label{tab:hls-solution-7-loop1-pipeline-export-rtl-resoruce-usage}
	\end{minipage}
	\hfill
	\begin{minipage}[t]{0.45\linewidth}
		\centering
		\begin{tabular}{|l|r|}
			\hline
			\textbf{Timing} & \textbf{VHDL} \\
			\hline
			CP required & 10.000 \\
			\hline
			CP achieved post-synthesis & 5.745 \\
			\hline
			CP achieved post-implementation & 5.762 \\
			\hline
		\end{tabular}
		\caption{HLS Solution 9 with columnIndex, values and x partitioning and loop1 pipelined Export RTL Final Timing}
		\label{tab:hls-solution-7-loop1-pipeline-export-rtl-final-timing}
	\end{minipage}
\end{table}